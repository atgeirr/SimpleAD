\documentclass[UKenglish,10pt]{beamer}
\usepackage{amsmath, amsfonts, amssymb}
\usepackage{pgf, tikz, bbm, svn, qtree}
\usepackage[latin1]{inputenc}
\usepackage{listings}
\usepackage{color}
\usepackage{textcomp}
\usepackage[vlined,oldcommands]{algorithm2e}    % For those with brand new LaTeX
\usepackage{appendixnumberbeamer}
\usepackage{listings}

\usetikzlibrary{positioning}

\mode<presentation>
{
  \usetheme{Sintef}
%  \usetheme{Copenhagen}
}

\usefonttheme{professionalfonts}

\definecolor{sintefblue}{rgb}{0.0,0.2,0.4}
\definecolor{dr}{rgb}{0.6,0.0,0.0}
\definecolor{dg}{rgb}{0.0,0.3,0.0}
\definecolor{db}{rgb}{0.0,0.0,0.5}
\setbeamercolor{uppercol}{fg=white,bg=sintefblue!80}
\setbeamercolor{lowercol}{fg=orange,bg=black!15}
\setbeamertemplate{footline}[frame number]

\setbeamertemplate{blocks}[rounded][shadow=false]

\newenvironment<>{myblock}[1]{%
  \begin{actionenv}#2%
      \def\insertblocktitle{#1}%
      \par%
      \mode<presentation>{%
        \setbeamercolor{block title}{fg=white,bg=sintefblue}
       \setbeamercolor{block body}{fg=black,bg=blue!5}
       %\setbeamercolor{itemize item}{fg=orange!20!black}
       \setbeamertemplate{itemize item}[triangle]
     }%
      \usebeamertemplate{block begin}}
    {\par\usebeamertemplate{block end}\end{actionenv}}

\newcommand{\COto}{CO\ensuremath{_\mathsf{2}}}
\newcommand{\Tensor}[1]{\ensuremath{\mathsf{#1}}}
\newcommand{\Vector}[1]{\ensuremath{\boldsymbol{#1}}}
\newcommand{\Matrix}[1]{\ensuremath{\boldsymbol{#1}}}
\newcommand{\T}        {\ensuremath{\mathsf{T}}}
\newcommand{\Grad}     {\ensuremath{\nabla}}

\newcommand{\dunemod}[1]{\textsl{dune-{#1}}}
\newcommand{\opmmod} [1]{\textsl{opm{#1}}}
\newcommand{\dumux}    {DuMu\ensuremath{^{\mathrm X}}}

\newcommand{\code}[1]{\texttt{#1}}
\newcommand{\Code}[1]{\texttt{#1}}
\newcommand{\red}[1]{\color{red} #1}

\newcommand{\Kb}{{\bf K}}
\newcommand{\vb}{{\bf v}}
\newcommand{\gb}{{\bf g}}
\newcommand{\nb}{{\bf n}}
\newcommand{\xb}{{\bf x}}
\newcommand{\qb}{{\bf q}}
\newcommand{\ub}{{\bf u}}

\DeclareMathOperator{\Div}{div}
\DeclareMathOperator{\adj}{adj}

\newlength{\arrgh}
\settowidth{\arrgh}{0.4121}



\definecolor{listinggray}{gray}{0.9}
\definecolor{lbcolor}{rgb}{0.95,0.95,0.95}
\definecolor{lightgray}{gray}{0.3}
\lstset{%
  xleftmargin= 4pt,
  xrightmargin=4pt}
\lstdefinelanguage{MRST}{%
  alsoletter={...},%
  morekeywords={%                             % keywords
  break,case,catch,continue,elseif,else,end,for,function,global,%
  if,otherwise,persistent,return,switch,try,while,...},%
  comment=[l]\%,%                             % comments
  morecomment=[l]...,%                        % comments
  morestring=[m]',%                           % strings
}[keywords,comments,strings]%
\lstset{
  backgroundcolor=\color{lbcolor},
  tabsize=4,
  rulecolor=,
  language=MRST,
  basicstyle=\small,
  upquote=true,
  aboveskip={0.5\baselineskip},
  columns=flexible,
  showstringspaces=false,
  extendedchars=true,
%  breaklines=true,
  prebreak = \raisebox{0ex}[0ex][0ex]{\ensuremath{\hookleftarrow}},
  frame=single,
  showtabs=false,
  showspaces=false,
  showstringspaces=false,
  identifierstyle=\ttfamily,
  keywordstyle=\color[rgb]{0,0,1},
  commentstyle=\color[rgb]{0.133,0.545,0.133},
  stringstyle=\color[rgb]{0.627,0.126,0.941},
}
\newcommand{\mcode}[1]{\lstinline|#1|}

\tikzset{Workflow Stage/.style=%
        {rectangle, draw=blue!50, fill=blue!20, thick, %
         text width=35mm}}


\title{A simple introduction to automatic differentiation (AD)}
\author[Atgeirr F. Rasmussen]{Atgeirr Fl{\o} Rasmussen}
\institute[SINTEF]{SINTEF Digital, Mathematics and Cybernetics \\ \includegraphics[height=2cm]{figs/sintef_logo-400x4001}}

%%%%%%%%%%%%%%%%%%%%%%%%%%%%%%%%%%%%%%%%%%%%%%%%%%%%%%%%%%%%%%%%%%%%%%%%%%%%%%

\begin{document}

\begin{frame}
  \titlepage
\end{frame}



%============================================
\section{Why use automatic differentiation}
%============================================

%--------------------
\begin{frame}
  \frametitle{Availability of materials}

  The slides and programming examples are available on GitHub:

  \bigskip

  \url{https://github.com/atgeirr/SimpleAD}


\end{frame}

%---------------------------------
\begin{frame}[fragile]
  \frametitle{What does AD provide}
  \begin{center}
    \begin{tikzpicture}[auto]
      \matrix[column sep=15mm,row sep=7mm, ampersand replacement=\&]
      {
        \node [Workflow Stage] (f-proglang)  {\verb|f(x) { ... }|}; \&
        \node [Workflow Stage] (df-proglang) {\verb|df(x) { ... }|}; \\
        \node [Workflow Stage] (f-analytic)  {$f(x)$}; \&
        \node [Workflow Stage] (df-analytic) {$f'(x)$}; \\
      };
      \draw<2-> [->]             (f-analytic) -- (df-analytic);
      \draw<2-> [->]             (f-analytic) -- (f-proglang);
      \draw<2-> [->]             (df-analytic)-- (df-proglang);
      \draw<3-> [->,thick,green] (f-proglang) -- (df-proglang);
    \end{tikzpicture}
  \end{center}

  \begin{myblock}<2->{Traditional Process}
    \begin{itemize}
    \item Human implements code to evaluate $f(x)$
    \item Manual or symbolic calculation to derive $f'(x)$
    \item Human implements code to evaluate $f'(x)$
    \end{itemize}
  \end{myblock}

  \begin{myblock}<3->{Automatic Differentiation (AD)}
    \begin{itemize}
    \item Human implements code to evaluate $f(x)$
    \item Computer code to evaluate $f'(x)$ is automatically generated
    \end{itemize}
  \end{myblock}
\end{frame}

%--------------------
\begin{frame}
  \frametitle{Benefits of using AD}
  AD makes it easier to create simulators:
  \begin{itemize}
  \item only specify nonlinear residual equation
  \item automatically evaluates Jacobian
  \item sparsity structure of Jacobian automatically generated
  \end{itemize}
  \bigskip

  Note that AD is {\em not} the same as finite differencing!
  \begin{itemize}
  \item no need to define a 'small' epsilon
  \item as precise as hand-made Jacobian
  \item ... but much less work!
  \end{itemize}
  \bigskip

  Performance (of equation assembly) may be slower than a
  {\em good} hand-made Jacobian implementation.
  \begin{itemize}
  \item We have also seen the opposite
  \item The more complex the equations, the better is AD
  \end{itemize}

\end{frame}

%--------------------
\begin{frame}
  \frametitle{Basic idea}
  A numeric computation $y =  f(x)$ can be written ($D$ = derivative)
  \begin{align*}
    y_1 & = f_1(x)    & \frac{dy_1}{dx}(x) & = Df_1(x) \\
    y_2 & = f_2(y_1) & \frac{dy_2}{dx}(x) & = Df_2 (y_1) \cdot Df_1(x) \\
    & \vdots \\
    y & = f_n(y_{n-1}) & \frac{dy}{dx}(x)  & = Df_n(y_{n-1}) \cdot
    Df_{n-1}(y_{n-2}) \cdots Df_1(x) \\
  \end{align*}
  Automatic Differentiation:
  \begin{itemize}
  \item make each line an elementary operation
  \item compute right derivative values as we go using chain rule
  \end{itemize}
\end{frame}

%--------------------
\begin{frame}
  \frametitle{Implementation approaches}
  Two main methods:
  \begin{myblock}{Operator overloading}
  \begin{itemize}
  \item requires operator overloading in programming language
  \item syntax (more or less) like before (non-AD)
  \item efficiency can vary a lot, depends on usage scenario
  \item easy to implement and experiment with
  \item Examples: OPM, MRST, Sacado (Trilinos), ADOL-C
  \end{itemize}
  \end{myblock}
  \begin{myblock}{Source transformation with AD tool}
  \begin{itemize}
  \item can be implemented for almost any language
  \item may restrict language syntax or features used
  \item efficiency can be high (depends on AD tool)
  \item Examples: TAPENADE, OpenAD
  \end{itemize}
  \end{myblock}
\end{frame}

%--------------------
\begin{frame}
  \frametitle{Types of AD}
  Two different approaches. \\
  (We compute $f(x)$, $u$ is some intermediate variable.)

  \begin{myblock}{Forward Mode}
    Carry derivatives with respect to independent variables: $$(u, \frac{du}{dx})$$
 \end{myblock}

  \begin{myblock}{Reverse Mode}
    Carry derivatives with respect to dependent variables (adjoints): $$(u, \frac{df}{du})$$
 \end{myblock}

\end{frame}

%============================================
%\section{Forward Mode AD}
%============================================

% %---------------------------------
% \begin{frame}
%   \frametitle{Overview of talk}
%   \tableofcontents[currentsection]
% \end{frame}

%--------------------
\begin{frame}
  \frametitle{Forward AD example (1)}
  Example function: $f(x) = x(sin(x^2) + 3x)$.

  Sequence of elementary functions:
  \begin{align*}
    f_1(u) & = u^2 & f'_1(u) & = 2uu' \\
    f_2(u) & = \sin(u)     & f'_2(u) & = \cos(u)u' \\
    f_3(u) & = 3u            & f'_3(u) & = 3u' \\
    f_4(u, v) & = u + v    & f'_4(u,v) & = u' + v' \\
    f_5(u, v) & = u \cdot v    & f'_5(u,v) & = u'v + uv' \\
  \end{align*}

  \begin{columns}[T]
    \column{0.5\textwidth}
    Rewritten: $f(x) = f_5(x, f_4(f_2(f_1(x)), f_3(x)))$
    \column{0.5\textwidth}
    \Tree [.$f_5$ $x$ [.$f_4$ [.$f_2$ [.$f_1$ $x$ ] ] [.$f_3$ $x$ ] ] ]
  \end{columns}
\end{frame}

%--------------------
\begin{frame}
  \frametitle{Forward AD example (2)}
  Example function: $f(x) = x(sin(x^2) + 3x)$.
  Computing $f(3), f'(3)$.
  Sequence of elementary functions:
  \begin{align*}
    f_1(u) & = u^2 & f'_1(u) & = 2uu' \\
    f_2(u) & = \sin(u)     & f'_2(u) & = \cos(u)u' \\
    f_3(u) & = 3u            & f'_3(u) & = 3u' \\
    f_4(u, v) & = u + v    & f'_4(u,v) & = u' + v' \\
    f_5(u, v) & = u \cdot v    & f'_5(u,v) & = u'v + uv' \\
  \end{align*}

  \begin{columns}[c]
    \column{0.5\textwidth}
    \Tree
    [.{\only<1-5>{$f_5$} \only<6>{\red $f_5$} \only<7->{28.2364}}
    {\only<1>{\red $x$} \only<2->{3}}
    [.{\only<1-4>{$f_4$} \only<5>{\red $f_4$} \only<6->{9.4121}}
    [.{\only<1,2>{\makebox[\arrgh][c]{$f_2$}} \only<3>{\red \makebox[\arrgh][c]{$f_2$}} \only<4->{0.4121}}
    [.{\only<1>{$f_1$} \only<2>{\red $f_1$} \only<3->{9}}
    {\only<1>{\red $x$} \only<2->{3}}
    ] ]
    [.{\only<1-3>{$f_3$} \only<4>{\red $f_3$} \only<5->{9}}
    {\only<1>{\red $x$} \only<2->{3}}
    ] ] ]
    \column{0.5\textwidth}
    \Tree
    [.{\only<1-5>{$f'_5$} \only<6>{\red $f'_5$} \only<7->{2.0118}}
    {\only<1>{\red $x'$} \only<2->{1}}
    [.{\only<1-4>{$f'_4$} \only<5>{\red $f'_4$} \only<6->{-2.4668}}
    [.{\only<1,2>{\makebox[\arrgh][c]{$f'_2$}} \only<3>{\red \makebox[\arrgh][c]{$f'_2$}} \only<4->{-5.4668}}
    [.{\only<1>{$f'_1$} \only<2>{\red $f'_1$} \only<3->{6}}
    {\only<1>{\red $x'$} \only<2->{1}}
    ] ]
    [.{\only<1-3>{$f'_3$} \only<4>{\red $f'_3$} \only<5->{3}}
    {\only<1>{\red $x'$} \only<2->{1}}
    ] ] ]
  \end{columns}
\end{frame}

%--------------------
\begin{frame}
  \frametitle{Properties of forward AD}
  \begin{itemize}
    \item Easy to implement with operator overloading
    \item Storage required (scalar): $2 \times$ normal (value, derivative).
    \item Storage required ($f: R^m \to R^n$): $(n+1) \times$ normal
      (value, derivative vector), unless sparse.
  \end{itemize}
\end{frame}


%---------------------------------
\begin{frame}
  \frametitle{Automatic Differentiation: OPM implementations}

  \begin{myblock}{\code{class Evaluation}}
    \begin{itemize}
    \item class implementing {\em forward AD} 
    \item deals with {\em a single scalar value} at a time
    \item derivatives are {\em compile-time-size vectors}
    \item implemented with operator overloading
    \item discrete div, grad must be implemented ``manually''
    \end{itemize}
  \end{myblock}

  \begin{myblock}{Dealing with off-diagonal Jacobian entries}
    \begin{itemize}
    \item Create the non-zero structure of the Jacobian upfront
    \item Focus on one cell's variables at a time
    \item Compute all terms depending on these
    \item Accumulate derivatives in correct Jacobian column
    \item Requires calculating fluxes twice
    \begin{itemize}
    \item Value the same, but derivatives are different \\
      (w.r.t. each neighbour cell)
    \end{itemize}
    \end{itemize}
  \end{myblock}

\end{frame}

%---------------------------------
\begin{frame}[fragile]
  \frametitle{A simple (forward) AD example class}

\begin{lstlisting}[language=c++]
class SimpleAd
{
private:
    double val_; // The value (corresponding to a regular double)
    double der_; // The derivative (of this variable)
 public:
    SimpleAd(double val, double der) : val_(val), der_(der) {}
    double value()      const { return val_; }
    double derivative() const { return der_; }
    SimpleAd operator+(const SimpleAd& rhs) const
    {
        // Derivative of sum is sum of derivatives.
        return { val_ + rhs.val_, der_ + rhs.der_ };
    }
    SimpleAd operator*(const SimpleAd& rhs) const
    {
        // Derivative of product follows well-known product rule.
        return { val_ * rhs.val_, der_ * rhs.val_ + val_ * rhs.der_ };
    }
};
\end{lstlisting}


\end{frame}

%---------------------------------
\begin{frame}
  \frametitle{Exercises (part 1: making do without AD)}

  These exercises assume the file newtonexample-exercise.cpp is available.

  \begin{enumerate}
  \item The example file contains a simple Newton's method that requires both
    a function and its derivative to be passed. Read and understand the
    function \code{newtonUpdate()}. {\em
      Compile and run the example, verifying that it produces the expected result.}
  \item Changing the function, you must also change the derivative function. {\em
      Uncomment ``part 2'' and fix the derivative until the example again
    compiles and runs successfully.}
  \end{enumerate}

\end{frame}



%---------------------------------
\begin{frame}
  \frametitle{Exercises (part 2: using AD)}

  These exercises assume the file adexample-exercise.cpp is available.

  \begin{enumerate}
  \item The example file contains a simple Newton's method that uses AD for
    its implementation. Read and understand \code{newtonUpdate()}. {\em
      Compile and run the example, verifying that it produces the expected result.}
  \item Functions such as sine, cosine and exponential require special
    treatment. {\em Using the provided sin() function as an example,
      implement a cos() function.}
  \item There are still operators missing for more general expressions. {\em
      Uncomment ``part 2'' and add missing features until the example again
    compiles and runs successfully.}
  \item An AD class must handle expressions containing raw doubles, either
    by a) including appropriate operators, or b) implicit conversions to the AD
    type. {\em What approach has been used for SimpleAD in the example file? How
      would the alternative approach have been implemented?}
  \end{enumerate}

\end{frame}




%---------------------------------
\begin{frame}
  % \frametitle{}

  \begin{centering}

  {\bf Thank you for listening!}
  \end{centering}
\end{frame}





%--------------------
\begin{frame}
  \frametitle{Bonus: Reverse Mode AD}
  \begin{myblock}{Recall: Reverse Mode}
    Carry derivatives with respect to dependent variables (adjoints): $$(u, \frac{df}{du})$$
 \end{myblock}

 We will use the chain rule again, but in the opposite direction:
 $$
 \adj(u) = \adj(f_i) \frac{\partial f_i}{\partial u}.
 $$

 Using $\adj(u)$ to mean the adjoint $\frac{df}{du}$. \\ (So $\adj(x)$ is our goal.)
\end{frame}

%--------------------
\begin{frame}
  \frametitle{Reverse AD example}
  Example function: $f(x) = x(sin(x^2) + 3x)$.
  Computing $f(3), f'(3)$.
  Sequence of elementary functions:
  \begin{align*}
    f_1(u) & = u^2             & \adj(u) & = \adj(f_1) \cdot 2 u \\
    f_2(u) & = \sin(u)        & \adj(u) & = \adj(f_2) \cdot \cos(u) \\
    f_3(u) & = 3u               & \adj(u) & = \adj(f_3) \cdot 3 \\
    f_4(u, v) & = u + v       & 
    \adj(u) & = \adj(f_4), & \adj(v) & = \adj(f_4) \\
    f_5(u, v) & = u \cdot v &
    \adj(u) & = \adj(f_5)\cdot v, & \adj(v) & = \adj(f_5)\cdot u \\
  \end{align*}

  \vspace{-1em}

  \begin{columns}[c]
    \column{0.3\textwidth}
    \Tree
    [.28.2364
    3
    [.9.4121
    [.0.4121
    [.9
    3
    ] ]
    [.9
    3
    ] ] ]
    \column{0.4\textwidth}
    \Tree
    [.{\only<1>{$\adj(f_5)$} \only<2>{\red $\adj(f_5)$} \only<3->{1}}
    {\only<1-6>{$\adj(x)$} \only<7>{\red $\adj(x)$} \only<8->{9.4121}}
    [.{\only<1,2>{$\adj(f_4)$} \only<3>{\red $\adj(f_4)$} \only<4->{3}}
    [.{\only<1-3>{$\adj(f_2)$} \only<4>{\red $\adj(f_2)$} \only<5->{3}}
    [.{\only<1-4>{$\adj(f_1)$} \only<5>{\red $\adj(f_1)$} \only<6->{-2.7334}}
    {\only<1-6>{$\adj(x)$} \only<7>{\red $\adj(x)$} \only<8->{-16.4003}}
    ] ]
    [.{\only<1-5>{$\adj(f_3)$} \only<6>{\red $\adj(f_3)$} \only<7->{3}}
    {\only<1-6>{$\adj(x)$} \only<7>{\red $\adj(x)$} \only<8->{9}}
    ] ] ]
    \column{0.3\textwidth}
    \visible<9>{Must sum contributions: $f'(3) = 9.4121 - 16.4003 + 9 = 2.0118$.}
  \end{columns}
\end{frame}

\end{document}


%%% Local Variables:
%%% fill-column: 76
%%% mode: LaTeX
%%% TeX-master: t
%%% End:
